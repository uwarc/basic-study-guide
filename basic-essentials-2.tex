\documentclass[letterpaper,12pt]{scrartcl}
\usepackage[utf8]{inputenc}
\usepackage{amsmath}
\usepackage{amssymb}

\title{Amateur Radio Basic Qualification -- The Essentials}
\subtitle{Section Two: Practical Operation}
\author{University of Waterloo Amateur Radio Club}
\date{\today}

\begin{document}

\maketitle
\tableofcontents

\section{Introduction}

These notes were prepared from Issue 3 of RIC-7 ``Basic Qualification Question Bank for Amateur Radio Operator Certificate Examinations'', published April 2007.
They cover 100\% of testable material on the Basic Qualification examination, but do not go beyond what is absolutely necessary to know in order to pass the examination.
The candidate is encouraged to perform their own research on topics that are not fully covered here.

\section{The Essentials: Section Two}

\subsection{Repeaters}

\begin{itemize}
\item A repeater is an autonomous amateur radio transceiver that receives a radio signal on one frequency (the ``input'' or ``uplink'' frequency)
and automatically retransmits it on another frequency (the ``output'' or ``downlink'' frequency).
\item Since repeaters are permanent installations with registered frequencies, they are usually a common meeting place on the air for local hams.
Repeaters are useful because they can re-transmit a weak signal with much more power than was used to send it. For this reason,
repeaters are mainly used to \textit{increase the range} of portable and mobile stations.
\item Some repeaters have an ``autopatch'', which is a device that allows repeater users to make telephone calls from their stations.
These are not seen as often on newer installations because of the prevalence of cell phones.
\item Most repeaters implement a ``time-out timer'' which limits the duration over which someone can transmit into the repeater continuously.
If transmission continues after the time-out, the repeater will temporarily stop re-transmitting that signal until it stops.
(This is useful if, for example, someone's push-to-talk button is stuck down.)
\item Most repeaters use a system known as CTCSS (Continuous Tone-Coded Squelch System) or PL (Privacy Line) to filter unwanted interference
on the repeater's input frequency. A simple explanation of PL tones is that every transmission into a PL-using repeater
must have a certain predefined sub-audible tone added to the audio signal in order for the repeater to accept (and retransmit) that signal.
For example, the VE3KSR repeater in the Waterloo region uses a PL tone of 131.8~Hz. If you transmit into the repeater without providing that tone
on your signal, you won't make it through the repeater.
\item Since repeaters are shared systems, you should pause briefly between transmissions when using a repeater to listen for anyone else wanting to use it.
For the same reason, you should keep transmissions reasonably short because longer transmissions may prevent someone with an emergency from using the repeater and getting help.
\item If you are trying to contact a specific individual on a repeater, say the call sign of the station you want to contact, then say your call sign.
\item If you want to join a conversation on a repeater, wait for a break between transmissions and then say your call sign.
\item The proper way to ask someone their location when using a repeater is simply to ask ``Where are you?''.
Using CB radio slang like ``What's your 20?'' is universally discouraged, and you will sound like a fool.
\item FM repeater operation on the two-meter band (144 -- 148~MHz) separates the repeater's input and output frequencies
by 600~kHz. For example, the VE3KSR repeater in the Waterloo region uses 146.970~MHz as the output frequency and 146.370~MHz as the input frequency.
\end{itemize}

\subsection{Standard International Phonetics}

To make your call sign better understood, and to spell words letter-by-letter when using voice transmissions,
you can use Standard International Phonetics (also known as the NATO Phonetic Alphabet).
You should learn this table and practice it in order to be a good operator who can use and recognize NATO phonetics on the air.
(This will also help you off the air if you ever need to spell your email address over the telephone!)

\begin{tabular}{c|l}
Letter & Phonetic letter \\
\hline 
A&Alpha \\
B&Bravo\\
C&Charlie\\
D&Delta\\
E&Echo\\
F&Foxtrot\\
G&Golf\\
H&Hotel\\
I&India\\
J&Juliet\\
K&Kilo\\
L&Lima\\
M&Mike\\
N&November\\
O&Oscar\\
P&Papa\\
Q&Quebec\\
R&Romeo\\
S&Sierra\\
T&Tango\\
U&Uniform\\
V&Victor\\
W&Whiskey\\
X&X-ray\\
Y&Yankee\\
Z&Zulu
\end{tabular}

Numbers are pronounced digit by digit in the usual way, with the caveat that the number 9 is prounced ``nine-er'' to distinguish it from ``five''.

\subsection{Simplex Operation}

\begin{itemize}
\item Simplex operation is transmitting and receiving on the same frequency.
(Contrast with repeater operation, which transmits on one frequency and receives on another.)
\item You should use simplex operation instead of a repeater when and if contact is possible without using a repeater.
If for any reason you find yourself trying to operate simplex on a repeater frequency, it is a good idea to change your frequency
because asking the repeater to change frequency is not practical.
\item If you are talking to a station using a repeater, one thing to try is listening for that station on the repeater's \textit{input} frequency.
If you can copy the station clearly, it might be possible to communicate using simplex instead.
\item It is possible to work simplex on VHF/UHF frequencies as well as HF frequencies.
Because VHF and UHF frequencies are not usually capable of long-distance communications, it is recommended to reserve them for local calls
and only use HF when it is desirable to comunicate over a longer distance. This reduces the interference on HF bands.
\item If you want to use a simplex frequency, listen first so that you do not interrupt a communication already in progress.
\item To put out a general call to ``any listening station'' is known as ``calling CQ''. (This is probably because saying the letters C-Q sounds like the phrase ``seek you'',
like you are looking for someone.) To call CQ on voice, say ``CQ'' three times, followed by ``this is'', followed by your call sign spoken three times.
To respond to a voice CQ call, say the other station's call sign once, then ``this is'', then your call sign given phonetically.
\item Here's an example of that last one. Suppose I (VE3TUX) want to see if I can reach anyone on simplex.
I start by picking a clear frequency, listening for a bit to make sure it really is free, and then saying 
``C Q C Q C Q this is Victor Echo Three Tango Uniform X-Ray Victor Echo Three Tango Uniform X-Ray Victor Echo Three Tango Uniform X-Ray''.
Now suppose my friend Peter (VA3VCF) hears my call and wants to respond. He will say ``Victor Echo Three Tango Uniform X-Ray this is Victor Alpha Three Victor Charlie Foxtrot''
and now that our handshake is complete we can have a nice conversation.
Remember to give your callsign at least every 30 minutes if you continue talking for that long.
A nice thing I might say when I am done is ``Seven Three'' (73), which is short for ``best wishes''.
To end the conversation properly, I could say ``Victor Echo Three Tango Uniform X-Ray signing off'' or something like that, if I am going to stick around for a while,
or ``Victor Echo Three Tango Uniform X-Ray clear / shutting down'' if I am turning off my radio.
\item There is a mode of operation popular for voice called Single Sideband (SSB). You will learn more about it a bit later on, but
it is most popular in the HF bands and rarely heard elsewhere. An important thing to know before you make or answer a call is which sideband to use.
Typically, lower sideband (LSB) is used on the 40m and 80m bands (7~MHz and 3.5~MHz), and upper sideband (USB) is used elsewhere. This is primarily for historical reasons,
but has become convention.
\item The best way to tell if a certain band is ``open'' for communication with a particular distant location is to listen for signals from that area
from an amateur beacon station or a foreign broadcast or television station on a nearby frequency.
\end{itemize}

\subsection{Minimizing Interference}

\begin{itemize}
\item It bears repeating, but before you transmit on any frequency, listen to make sure others are not already using the frequency.
\item One way to shorten transmitter tune-up time on the air to cut down on interference is to tune into a ``dummy load'', which is 
an electrical device that does not radiate RF energy but has similar electrical properties to an antenna.
\item If you contact another station and your signal is reported as extremely strong and perfectly readable,
you may be able to reduce your power output to the minimum necessary. This reduces interference to others who may be operating nearby.
\item Occasionally groups of hams may decide to meet on a pre-coordinated frequency at a specified time.
This is known as a ``net''. If you are the net control station (coordinator) of a daily HF net,
and you find the net frequency is in use just before the net begins,
you should conduct the net on a frequency 3 to 5~kHz away from the regular net frequency.
Conversely, if a net is about to begin on a frequency that you are using, as a courtesy to the net you should move to a different frequency.
\item If propagation changes during your contact and you notice other activity nearby increasing interference, move your contact to a different frequency.
\item To avoid interfering with other stations when operating on single sideband, allow a minimum frequency separation of 3~kHz from a contact in progress to minimize interference.
\end{itemize}

\subsection{CW (Morse) Operation}

\begin{itemize}
\item You should transmit on Morse at any speed which you can reliably receive.
\item The procedural signal ``CQ'' means ``calling any station''.
\item The procedural signal ``DE'' means ``from'' or ``this is''.
\item The correct way to call CQ on Morse is to send the letters ``CQ'' three times, followed by ``DE'', then your call sign sent three times.
\item To respond to a CQ call on Morse is to send the other station's call sign twice, followed by ``DE'', followed by your call sign twice.
\item The procedural signal ``K'' means ``any station transmit''.
\item The term ``DX'' means ``distant station''.
\item The term ``73'' means ``best regards'' (also heard on voice as ``seven three'').
\item ``Full break-in telegraphy'' means receiving incoming signals between outgoing Morse dots.
\item When selecting a CW transmitting frequency, keep a minimum frequency separation of 150 to 500~Hz from a contact in progress.
\item Good Morse telegraphy operators listen to the frequency to make sure that it is not in use before transmitting.
\end{itemize}

\subsection{Signal Reports}

\begin{itemize}
\item ``RST'' signal reports are a short way to describe signal reception.
``RST'' stands for ``readability'', ``signal strength'', and ``tone''.
The signal report is given as a two-digit code (corresponding to R and S)
or a three-digit code (corresponding to R, S, and T).
\item The code for R goes from 1 to 5, where 1 means ``unreadable'' and 5 means ``perfectly readable''.
\item The code for S goes from 1 to 9, where 1 means ``faint signal'' and 9 means ``very strong''.
\item The code for T (only given during CW or digital) goes from 1 to 9, where 1 means ``very rough tone'' and 9 means ``perfect tone''.
\item The ``S'' in RST is usually measured with an S meter, or a relative signal-strength indicator.
If a signal report is given ``9 plus $N$ dB'', it means that the reading is 20 decibels greater than strength 9 (which is an extremely strong signal).
\item An increase in transmitter power of four times corresponds to an increase of approximately one S unit on an S meter.
\end{itemize}

\subsection{Q Signals}

Q signals are short three-letter abbreviations that all begin with the letter Q. They save time on CW and digital communications.
Think of them as the ``SMS speak'' of the ham radio world. And now you know why they are a problem.

I'm really sorry about this part of the exam. This segment is pure memorization, even worse than anything else so far.
The meaning of these signals is not at all obvious just by looking at or hearing them, and what's worse is that for the most part
you won't encounter these unless you are operating CW (or very rarely on some digital modes)
because they provide no benefit on voice. I'll spare you a lot of trouble by only putting the ones you will actually be tested on,
and you can look up the rest yourself if you encounter them.

One thing to know is that you can put a question mark after any Q-signal to turn it into a question.
For example, ``QRS?'' means ``Should I send more slowly?''. A response of ``QRS'' means ``Send more slowly.''.

\begin{itemize}
\item QRS -- send more slowly
\item QTH -- my location is
\item QRL? -- is this frequency in use?
\item QSY -- change frequency
\item QSO -- contact in progress (this one is unfortunately somewhat common off the air as well, as in
``I had a really nice cue-soh with John yesterday evening'')
\item QRZ? -- is someone calling me?
\item QRM -- I am being interfered with
\item QRN -- I am troubled by static
\item QRX -- I will call you again
\item QSL -- I acknowledge receipt (most commonly seen with ``QSL cards'')
\end{itemize}

\subsection{Emergency Operation}

\begin{itemize}
\item If you are in contact with another station and you hear an emergency call for help on your frequency,
immediately stop your contact and take the emergency call. Acknowledge the station in distress, determine its location,
and find out what assistance may be needed. If you cannot render assistance, you should maintain watch until you are
certain that assistance will be forthcoming.
\item When operating voice, the proper distress call to use is ``MAYDAY''.
\item When operating CW, the proper distress call to use is ``SOS''.
\item You may only transmit ``MAYDAY'' or ``SOS'' in a life-threatening distress situation.
\item The proper way to interrupt a repeater conversation to signal a distress call is to say ``BREAK'' twice, followed by your call sign.
\item It is a good idea to have a way to operate your amateur statio without using commercial power lines
(e.g. battery-powered handheld, generator) so you may provide communications in an emergency.
\item The most important accessory to have for a hand-held radio in an emergency is a charged spare battery -- preferably several.
\item A dipole antenna is small, simple, and fast to set up and store. It is a good choice as part of a portable HF station that could be
assembled in an emergency.
\item In order of priority, a distress message comes before an urgency message, which comes before a safety message.
\end{itemize}

\subsection{International Operation}

\begin{itemize}
\item A ``QSL card'' is a written proof of communication between two amateurs. They are typically exchanged for long-distance contacts made over HF.
\item An azimuthal map is a map projection centered on a particular location that is used to determine the shortest path between points on the earth's surface.
It is useful when orienting a directional HF antenna toward a distant station.
\item When orienting a directional antenna, it may be pointed along the ``short path'', which is directly at the remote station,
or along the ``long path'', which is 180 degrees away from the remote station.
If you cannot hear a station on the short path, it may be worth trying the long path to see if propagation is different.
\item A station logbook is important for recording contacts for operating awards, and very important for handling neighbour interference complaints.
A well-kept log preserves your fondest amateur radio memories for years (this quoted verbatim from the question bank).
However, Industry Canada does not formally require any operator to keep a logbook.
\item Station logs and QSL cards are always kept in UTC (Coordinated Universal Time). This time is based in Greenwich, England (formerly Greenwich Mean Time, GMT).
\item To set your station clock accurately to UTC, you could receive the most accurate time off the air from
a time station such as CHU, WWV, or WWVH.
\end{itemize}

\end{document}

